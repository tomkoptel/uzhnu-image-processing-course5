%% Преамбула TeX-файла

% 1. Стиль и язык
\documentclass[utf8x, 12pt]{G7-32} % Стиль (по умолчанию будет 14pt)

% Остальные стандартные настройки убраны в preamble.inc.tex.
\include{10-preamble}

%Code-specific snippets
%--------------------------------------
\usepackage{listings}
\usepackage{xcolor}

\definecolor{codegreen}{rgb}{0,0.6,0}
\definecolor{codegray}{rgb}{0.5,0.5,0.5}
\definecolor{codepurple}{rgb}{0.58,0,0.82}

\lstdefinestyle{light}{
commentstyle=\color{codegreen},
keywordstyle=\color{magenta},
numberstyle=\tiny\color{codegray},
stringstyle=\color{codepurple},
basicstyle=\ttfamily\footnotesize,
breakatwhitespace=false,
breaklines=true,
captionpos=b,
keepspaces=true,
numbers=left,
numbersep=5pt,
showspaces=false,
showstringspaces=false,
showtabs=false,
tabsize=2
}

\lstset{style=light}
%--------------------------------------

\begin{document}

    \frontmatter % выключает нумерацию ВСЕГО; здесь начинаются ненумерованные главы: реферат, введение, глоссарий, сокращения и прочее.
    \begin{center}

    \large Міністерство освіти і науки України\\
    ДВНЗ «Ужгородський національний університет»\\
    Факультет інформаційних технологій\\
    Кафедра інформаційних управляючих систем та технологій\\
    \\[5.5cm]

    \huge Лабороторна робота \\[0.6cm] % название работы, затем отступ 0,6см
    \large на тему:  {Прогнозування \linebreak ринкової вартості}\\[3.7cm]


\end{center}

\begin{flushright}
    Виконував:\\
    студент V курсу\\
    напрямку підготовки\\
    <<Комп’ютерні науки та інформаційні технології>> \\
    Коптель Артем Олегович \\
    Перевірив: Ніколенко Володимир Володимирович
\end{flushright}


\vfill

\begin{center}
    \large Ужгород 2020
\end{center}

\thispagestyle{empty}


    \thispagestyle{empty}
    \setcounter{page}{0}
    \tableofcontents
    \clearpage


    \Introduction

Ми почнемо з огляду того, як працюють моделі машинного навчання та як вони використовуються.
Це може здатися базовим, якщо ви раніше займалися статистичним моделюванням чи машинним навчанням.

В даній роботі опрацьовується наступний сценарій:

Ваш двоюрідний брат заробив мільйони доларів спекулюючи на нерухомості.
Він запропонував стати з вами діловими партнерами за умовою, що він надає фінансування, кол ви моделі.
Ці моделі мають передбачувати вартість будинків спираючись на дані.
Вартість має максимально точно оцінювати ціни на різні будинки.
Чим краще модель тим краща передбачувана вартість, і в результаті ближчи до реальності ціна на нерухомість.

Ви запитаєте свого кузена, як він прогнозував вартість нерухомості в минулому і він каже, що це просто інтуїція.
Але додаткові опитування виявляють, що він визначив закономірності цін на будинки, які він бачив у минулому, і він використовує ці моделі, щоб робити прогнози щодо нових будинків, які він розглядає.
Машинне навчання працює по такому же принципу.

Ми почнемо з моделі, яка називається Деревом рішень.
Є вигадливіші моделі, які дають точніші прогнози.
Але дерева прийняття рішень легко зрозуміти, і вони є базовим елементом для найкращих моделей в галузі даних.

Для простоти ми почнемо з максимально простого дерева рішень.
Далі ми розглянемо питання оцінки результатів на значення середньої абсолютної помилки \textbf(MAE).
Оцінивши похибку ми спробуємо оптимізувати модель.
На останок, ми розглянемо альтернативну модель "випадковий ліс дерев" та порівняємо її ефективність з "деревом рішень".


    \mainmatter

    \chapter{Дерево рішень}\label{cha:decision_tree}
Ми використовуємо дані, щоб вирішити, як \underline{розбити} будинки на дві (див. Малюнок 1) \underline{групи}, а потім знову визначити прогнозовану ціну в кожній групі.
Цей етап фіксації шаблонів з даних називається \textbf{пристосуванням} або \textbf{навчанням} моделі.
Дані, що використовуються, щоб \textbf{відповідати} моделі, називаються \textbf{навчальними даними}.

Ви можете вловити більше факторів, використовуючи дерево, яке має більше «розбиттів».
Вони називаються \textbf{"глибшими" деревами}.
Дерево рішень, яке також враховує загальний розмір ділянки кожного будинку, може виглядати як на зображенні 2.

Прогнозована ціна на будинок знаходиться внизу дерева.
Точка внизу, де ми робимо прогноз, називається \textbf{листом}.

\begin{figure}
    \label{fig:image2}
    \centering
    \includegraphics[scale=0.5]{image2.png}

    Deeper Trees
\end{figure}

    \chapter{Оцінка даних}\label{cha:basic_data_exploration}
Першим кроком у будь-якому проекті машинного навчання є ознайомлення з даними.
Для цього ви будете використовувати бібліотеку Pandas.
Pandas - основний інструмент даних, який вчені використовують для вивчення та обробки даних.
Більшість людей скорочують панд у своєму коді як pd.
Ми робимо це за командою

\begin{lstlisting}[style=light, language=Python,label={lst:vectorimg},caption=Імпортування Pandas]
import pandas as pd
\end{lstlisting}

Найважливішою частиною бібліотеки Pandas є \textbf{DataFrame}.
DataFrame містить тип даних, яку ми можемо сприймати як таблицю.
Це схоже на аркуш у Excel або таблицю в базі даних SQL.

Як приклад, ми розглянемо дані про ціни на житло в Айові.

Приклади даних (Айові) знаходяться на шляху до файлу ../input/melbourne-housing-snapshot/train.csv.

Ми завантажуємо та досліджуємо дані за допомогою таких команд:

\begin{lstlisting}[style=light, language=Python,label={lst:vectorimg},caption=Імпортування Pandas]
import pandas as pd

# Path of the file to read
iowa_file_path = '../input/home-data-for-ml-course/train.csv'

# Fill in the line below to read the file into a variable home_data
c = pd.read_csv(iowa_file_path)

home_data = pd.read_csv(iowa_file_path)
home_data.describe()

\end{lstlisting}

\begin{figure}
    \label{fig:data_iowa}
    \includegraphics[scale=0.4]{data-iowa.png}

    Результат виконаня команди describe()
\end{figure}

\section{Опис даних}\label{sec:data_description}
Результати показують 8 чисел для кожного стовпця у вихідному наборі даних.
Перше число, \textbf{count}, показує, скільки рядків мають невідсутні значення.

Відсутні цінності виникають з багатьох причин.
Наприклад, розмір 2-ї спальні не буде збиратися при обстеженні 1-кімнатного будинку.
Ми повернемось до теми відсутніх даних.

Друге значення - \textbf{mean}, яке є середнім значенням.
Під цим, \textbf{std} - це стандартне відхилення, яке вимірює, наскільки чисельно розподілені значення.

Щоб інтерпретувати значення min, \textbf{25\%, 50\%, 75\%} та max, уявіть, як сортуватиме кожен стовпець від найнижчого до найвищого значення.
Перше (найменше) значення - це мін.
Якщо ми пройдемо чверть шляху по списку, ми знайдемо число, яке перевищує 25\% значень і менше 75\% значень.
Це значення 25\% (вимовляється "25-й процентиль").
50-й та 75-й процентилі визначаються аналогічно, і max - найбільша кількість.

\section{Інтерпретація}\label{sec:interpretation}

Найновштй будинок у наших даних не такий вже й новий.
Кілька можливих пояснень цьому:

\begin{itemize}
    \item В регіоні де збирали дані, не будували нових будинків.
    \item Дані були зібрані давно. Будинки, побудовані після публікації даних, не відображатимуться.
\end{itemize}

Скоріш за все, набір даних застарілий.
Ми можемо перевірити це, порівнявши сьогодні дані про житло в Айові з тими, що перелічені в наборі даних, і побачити останній рік побудови, рік продажу тощо.
Якщо будь-який із них перевищує 2011 рік, ми можемо підтвердити, що набір даних застарілий.

Це проблема, оскільки якщо ми використовуємо цей набір даних для прийняття рішень про купівлю, продаж та залучення на ринок житла в Айові, уявлення, які ми могли б отримати з цих даних, сьогодні, швидше за все, будуть не такими актуальними.

Приклад відмінностей:
\begin{itemize}
    \item Ціна продажу, ймовірно, вища через стабільно зростаючий ринок житла в США.
    \item Більше готових підвалів старих будинків.
    \item Різні особливості будинку можуть мати дещо іншу вагу, ніж 10 років тому.
\end{itemize}

    \chapter{Оптимізація моделі. Вибираємо оптимальне значення рівнів дерева}\label{cha:selecting_number_of_leafs}
Існує кілька альтернатив для управління глибиною дерева, і багато з них дозволяють, щоб деякі маршрути через дерево мали більшу глибину, ніж інші маршрути.
Але аргумент \textbf{max\_leaf\_nodes} забезпечує спосіб контролювати переобладнанням та недооснащенням облаштування.
Чим більше аркушів ми дозволяємо зробити моделі, тим більше ми переходимо від зони недообладнання на наведеному графіку до зони переоснащення.

Ми можемо використовувати функцію корисності, щоб допомогти порівняти оцінки MAE з різних значень для \textbf{max\_leaf\_nodes}.

З перелічених варіантів 500 - це оптимальна значення для нашої моделі.

\begin{lstlisting}[style=light, language=Python,label={lst:vectorimg},caption=Computing MAE for different value of leaf nodes]
from sklearn.metrics import mean_absolute_error
from sklearn.tree import DecisionTreeRegressor

def get_mae(max_leaf_nodes, train_X, val_X, train_y, val_y):
    model = DecisionTreeRegressor(max_leaf_nodes=max_leaf_nodes, random_state=0)
    model.fit(train_X, train_y)
    preds_val = model.predict(val_X)
    mae = mean_absolute_error(val_y, preds_val)
    return(mae)

# compare MAE with differing values of max_leaf_nodes
for max_leaf_nodes in [5, 50, 500, 5000]:
    my_mae = get_mae(max_leaf_nodes, train_X, val_X, train_y, val_y)
    print("Max leaf nodes: %d  \t\t Mean Absolute Error:  %d" %(max_leaf_nodes, my_mae))

# Max leaf nodes: 5  		 Mean Absolute Error:  347380
# Max leaf nodes: 50  		 Mean Absolute Error:  258171
# Max leaf nodes: 500  		 Mean Absolute Error:  243495
# Max leaf nodes: 5000       Mean Absolute Error:  254983
\end{lstlisting}

    \include{34-model-validation}
    \include{35-random-forest}

    \backmatter %% Здесь заканчивается нумерованная часть документа и начинаются ссылки и
    %% заключение

    \Conclusion % заключение к отчёту

Моделі можуть мати наступні недоліки:

\begin{itemize}
    \item \textbf{переобладнання}: захоплення помилкових патернів, які не повторяться в майбутньому, що призведе до менш точних прогнозів, або
    \item \textbf{недооснащення}: не вдалося вловити відповідні закономірності, що знову призводить до менш точних прогнозів.
\end{itemize}
Ми використовуємо дані перевірки, які не використовуються при навчанні моделей, для вимірювання точності кандидатської моделі.
Це дозволяє нам спробувати багато моделей кандидатів і, в кінці кінців, вибрати найкращу.

Є параметри, які дозволяють поліпшити нам ефективність випадкового лісу настільки, наскільки ми змінимо значення максимальної глибини одного дерева рішень.
Але однією з найкращих особливостей моделей Random Forest є те, що вони, як правило, прогнозують значення дуже добро навіть без додаткового налагоджування.


    \include{5-biblio}

\end{document}
