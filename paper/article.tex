\documentclass{article}
%Russian-specific packages
%--------------------------------------
\usepackage[T2A]{fontenc}
\usepackage[utf8]{inputenc}
\usepackage[russian]{babel}
%--------------------------------------

%Hyphenation rules
%--------------------------------------
\usepackage{hyphenat}
\hyphenation{ма-те-ма-ти-ка вос-ста-нав-ли-вать}
%--------------------------------------

\title{Обробка зображень - Василь Олександрович Лавер}
\author{Aртем Коптель }
\date{Листопад 2020}

\begin{document}

\maketitle

\tableofcontents

\section{Вступ}\label{sec:intro}

Цей документ збирається представити деякі методи цифрової обробки зображень, засновані на просторовій обробці,
які складають перетворення інтенсивності та просторову фільтрацію з використанням згладжуючих просторових фільтрів
та посилення просторових фільтрів.
Також будуть представлені результати цих методик.

\section{Загaльні відомості}\label{sec:general}

\subsection{Зображення}\label{subsec:image}
Зображення, визначене з математичної точки зору, вважається функцією двох реальних змінних, наприклад,
a (x, y) з а як амплітуда (наприклад, яскравість) зображення в реальній координатній позиції (x, у).
Крім того, можна вважати, що зображення містить підзображення, які іноді називають регіонами інтересів,
ROI або просто регіонами.
Ця концепція відображає той факт, що зображення часто містять колекції предметів, кожен з яких може бути основою для регіону.

\subsection{Цифрова обробка зображень}\label{subsec:image_preprocessing}
Це використання комп’ютерних інструментів для виконання деяких процесів на цифровому зображенні, ці інструменти часто
є комп’ютерним алгоритмом, що використовується для виконання певного завдання.
Як підполе цифрової обробки сигналів, цифрова обробка зображень має багато переваг перед аналоговою обробкою зображень;
це дозволяє застосовувати набагато ширший спектр алгоритмів до вхідних даних і дозволяє уникнути таких проблем,
як накопичення шуму та спотворення сигналу під час обробки.
\end{document}
