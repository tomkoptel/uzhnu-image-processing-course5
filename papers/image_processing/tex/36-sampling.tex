\chapter{Дискретизація або вибірка (Sampling)}\label{cha:sampling}
"Вибірка - це процес перетворення сигналу (наприклад, функції безперервного часу або простору) в числову послідовність (функція дискретного часу або простору).
Процес також називається аналого-цифровим перетворенням або просто оцифровкою"~\cite{wiki_sampling:2}.

Вимірюючи значення для пікселя, приймається середній колір області навколо місця розташування пікселя.
Спрощеною моделлю є вибірка квадрата, це називається \textbf{коробчатим фільтром}, більш фізично точне вимірювання полягає в обчисленні зваженого середнього за Гаусом значення.
Більше значення ваги набувається на координатах визначення пікселя, а менші значення ваги в областях навколо пікселів~\cite{gimp:1}.

При сприйнятті растрового зображення людське око "змішує" значення пікселів між собою, відтворюючи ілюзію безперервного зображення, яке воно представляє.
Іншими словами, око сприймає форму як різницю переходу між більшим та меньшим значення ваги пікселя (різниця між відтінками).

\section{Приклад дискретизації}\label{sec:example_of_sampling}
Зображення 256 х 256 пікселів, яке охоплює фізичну площу 100 х 100 мкм, має щільність вибірки 256/100 = 2,56 зразка на мікрон як вздовж X, так і Y.
Еквівалентно, розмір вибірки вздовж будь-якого з цих напрямків становить 100/256 ~ 0,391 мкм = 391 нм.
Чи достатньо цього для сучасних умов, визначається ідеальною частотою дискретизації~\cite{svi_sampling:3}.

Зміна коефіцієнта масштабування скануючого мікроскопа для сканування меншої фізичної площі, скажімо, 70 х 70 мкм, зберігаючи однакову кількість пікселів у записаному зображенні, 256 х 256 пікселів, дозволяє отримати роздільну здатність, зменшуючи розмір вибірки: 70 / 256 ~ 0,273 мкм = 273 нм.
Ми отримуємо більше зразків на певній фізичній відстані, дозволяючи розрізнити більше деталей.
Однак це обмежується дифракцією на лінзах, а вибірка, що перевищує ідеальну швидкість, не покращує ситуацію~\cite{svi_sampling:3}.

Мікроскопічному масштабуванню можна протиставити "цифрове збільшення": обрізання та інтерполяція отриманих даних.
Цифрове (після придбання) масштабування не забезпечує більше фізичної інформації або кращої роздільної здатності, структури не вирішуються краще.
Він споживає ресурси зберігання та пам'яті.
Дві різні функції, які виглядають як накладені через обмеження роздільної здатності, перекриуть одна іншу після цифрового масштабування~\cite{svi_sampling:3}.
